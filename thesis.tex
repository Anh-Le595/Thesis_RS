\documentclass[a4paper,12pt,numbered,print,index,custombib, oneside, custommargin]{report}
%\usepackage{ifpdf}
\newcommand\tab[1][1cm]{\hspace*{#1}}
%% Language and font encodings
\usepackage[utf8]{vietnam}
\usepackage[utf8]{inputenc}
\usepackage[english]{babel}


%% Sets page size and margins
\usepackage[a4paper,top=3cm,bottom=2cm,left=3cm,right=3cm,marginparwidth=1.75cm]{geometry}
\setlength{\parskip}{0.5em}
\setlength{\parindent}{0em}

%% Useful packages
\usepackage{amsmath}
\usepackage{amssymb}
\usepackage{graphicx}
\usepackage[colorinlistoftodos]{todonotes}
\usepackage[colorlinks=true, allcolors=blue]{hyperref}
\usepackage{subfigure}
\usepackage{float}
\usepackage{listings}
\usepackage{caption}
\usepackage{array}
\usepackage{color}
\usepackage{tabu}
\usepackage{longtable}
\usepackage{fancyhdr}
\usepackage{amsmath}


%New colors defined below
\definecolor{codegreen}{rgb}{0,0.6,0}
\definecolor{codegray}{rgb}{0.5,0.5,0.5}
\definecolor{codepurple}{rgb}{0.58,0,0.82}
\definecolor{backcolour}{rgb}{0.95,0.95,0.92}

%Code listing style named "mystyle"
\lstdefinestyle{mystyle}{
  backgroundcolor=\color{backcolour},   commentstyle=\color{codegreen},
  keywordstyle=\color{magenta},
  numberstyle=\tiny\color{codegray},
  stringstyle=\color{codepurple},
  basicstyle=\footnotesize,
  breakatwhitespace=false,         
  breaklines=true,                 
  captionpos=b,                    
  keepspaces=true,                 
  numbers=left,                    
  numbersep=5pt,                  
  showspaces=false,                
  showstringspaces=false,
  showtabs=false,                  
  tabsize=2
}

%"mystyle" code listing set
\lstset{style=mystyle}

\addto\captionsenglish{\renewcommand{\chaptername}{Chương}}
\addto\captionsenglish{\renewcommand{\contentsname}{Mục lục}}
\addto\captionsenglish{\renewcommand{\listfigurename}{Danh sách hình vẽ}}
\addto\captionsenglish{\renewcommand{\listtablename}{Danh sách bảng}}
\addto\captionsenglish{\renewcommand{\figurename}{Hình}}
\addto\captionsenglish{\renewcommand{\tablename}{Bảng}}
\addto\captionsenglish{\renewcommand{\bibname}{Tài liệu tham khảo}}

\title{HỆ THỐNG TƯ VẤN THÔNG MINH}
\author{Lê Tuấn Anh}

\begin{document}
\maketitle
\begin{center}
\Huge{LỜI CAM ĐOAN}
\end{center}
Tôi cam đoan toàn bộ nội dung của báo cáo đều do tôi nghiên cứu thực hiện và biên soạn, không sao chép từ bất kì tài liệu nào khác. Các thông tin tham khảo trong báo cáo đều được nêu rõ nguồn gốc. Tôi sẽ chịu trách nhiệm nếu có bất cứ sai phạm nào so với lời cam kết.


\tab \tab Tp. Hồ Chí Minh, tháng 12 năm 2017 \par
\tab \tab \tab \tab Lê Tuấn Anh
\pagebreak

\begin{center}
\Huge{LỜI CẢM ƠN}
\end{center}
Trong suốt thời gian làm việc và hoàn thành đề tài thực tập tốt nghiệp, Tôi xin gửi lời cảm ơn chân thành đến PGS. TS. Quản Thành Thơ, thầy đã giải đáp tận tình những thắc mắc của tôi và đồng thời giúp tôi định hướng cho đề tài của mình. Tôi cũng xin cảm ơn thầy Th.S Mai Đức Trung đã hỗ trợ tạo điều kiện cho tôi những buổi gặp mặt với thầy Thơ thuận tiện. Bên cạnh đó, Tôi xin cám ơn TS Vũ Quang Hiếu và TS Lê Duy Ngạn trong nhóm Data Science tại công ty Zalora, vì sự giúp đỡ nhiệt tình và những lời khuyên rất bổ ích đối với đề tài của tôi.
Tôi xin gửi lời cảm ơn tới tất cả các thầy cô trong khoa Khoa học và Kỹ thuật Máy tính đã hết lòng truyền dạy những kiến thức học thuật cũng như kinh nghiệm thực tế trong suốt quãng thời gian tôi học tập tại trường.
Tôi xin cảm ơn gia đình và bạn bè đã hỗ trợ, quan tâm và đồng hành cùng tôi trong suốt khoản thời gian 4 năm vừa qua.
Một lần nữa, tôi xin chân thành cảm ơn!


\tab \tab Tp. Hồ Chí Minh, tháng 12 năm 2017 \par
\tab \tab \tab \tab Lê Tuấn Anh
\pagebreak

\tableofcontents

\listoffigures

\listoftables


\chapter{Giới thiệu}

Trong chương này, tôi sẽ giới thiệu tổng quan về đề tài, mục tiêu đặt ra của đề tài ở giai đoạn Thực Tập Tốt Nghiệp và Luận Văn Tốt Nghiệp cũng như giới hạn của đề tài. Cuối cùng là phần giới thiệu cấu trúc tổng quát của báo cáo.

\section{Giới thiệu đề tài}
Từ những năm 90 của thế kỷ 20, hệ thống tư vấn đã trở thành đề tài hấp dẫn đối với nhiều công ty. Hiện nay, việc tiếp cận thông tin là việc rất dễ dàng cho mọi người khi họ sử dụng Internet. Từ đó, nó cũng tạo ra một cuộc bùng nổ thông tin toàn cầu. Và đặc biệt cùng với sự phát triển của Thương Mại Điện Tử (E-Commerce), số lượng thông tin được trao đổi tăng theo cấp số nhân. Vì vậy, thông tin sẽ được cung cấp nhanh chóng.
Tuy nhiên, sở thích của mỗi người là khác nhau. Vì thế, sẽ mất nhiều thời gian về việc tìm kiếm thông tin hoặc sản phẩm nào phù hợp với mục đích của mình. Việc sử dụng các công cụ tìm kiếm như Google, Bing, … có thể giúp chúng ta cải thiện việc tìm những thông tin cần thiết, nhưng với lượng thông tin ngày càng tăng, các công cụ tìm kiếm không thể đáp ứng hoàn toàn nhu cầu của con người, bởi vì thời gian bỏ ra vẫn khá lớn khi thông tin cần phải được lọc để nhận được thông tin chính xác cho vấn đề chúng ta muốn tìm kiếm. Hoặc do chúng ta có thể sẽ chưa xác định rõ được thông tin chúng ta muốn tìm kiếm là gì. 
Hệ Thống Tư Vấn (Recommendation Systems) ra đời. Hệ thống này sẽ tư vấn cho người dùng những thông tin cần thiết cho họ thông qua tính toán và dự đoán sở thích, mong muốn của họ. 
Từ đó sẽ tư vấn cho người dùng những thông tin phù hợp nhất cho họ. 
Với hệ thống tư vấn này sẽ làm tăng những trải nghiệm thú vị, chất lượng cho khách hàng, thu hút và tạo ra những khách hàng tìm năng, tăng khả năng tương tác với khách hàng. Từ đó, sẽ giúp sản phẩm của công ty đến gần hơn với khách hàng, chất lượng phục vụ từ đó sẽ được cải thiện.

Ví dụ: khi người dùng truy cập vào một trang web thương mại điện tử nào đó (Ví dụ: Zalora), giả sử khách hàng là nam và đang cần tìm một chiếc quần bò. Khi họ truy cập vào thông tin chi tiết của sản phẩm họ cần tìm, hệ thống tư vấn có nhiệm vụ gợi ý những sản phẩm khác hoặc đi kèm, hoặc phù hợp với sở thích của khách hàng, chẳng hạn áo, đồng hồ, mắt kính, thắt lưng… 
Ví dụ điển hình khác có thể là FaceBook. Khi một người mới đã có tài khoản, lần đầu tiên khi người này truy cập vào trang chủ của mình, họ sẽ được hệ thống tư vấn của FaceBook gợi ý kết bạn với những người bạn mới. Cách làm của hệ thống này có thể nó sẽ dựa vào những thông tin mà khi người này cung cấp khi đăng ký tài khoản, và thông qua mail của người này.

\section{Mục tiêu, giới hạn và các giai đoạn của đề tài}
\subsection{Mục tiêu và giới hạn của đề tài}
Mục tiêu của đề tài là xây dựng một hệ thống tư vấn người dùng khi họ tìm kiếm thông tin, sản phẩm, mặt hàng nào đó trên website: www.zalora.vn của công ty Zalora Việt Nam thông qua các phương pháp sẽ được đề cập trong bài báo cáo này.
Các thông tin này sẽ được hiển thị đồng thời với sản phẩm người dùng đang tìm kiếm và nội dung phù hợp với sự sở thích của khách hàng. 
Các thành phần của hệ thống cũng là sản phẩm của đề tài này như sau:

\subsection{Giai đoạn Thực Tập Tốt Nghiệp}
Mục tiêu của giai đoạn Thực Tập Tốt Nghiệp như sau:
\begin{itemize}
\item Tìm hiểu về Hệ Thống Tư Vấn.
\item Làm thế nào để xây dựng Hệ Thống Tư Vấn.
\item Các phương pháp tính toán khả năng người dùng ưa thích thông tin, sản phẩm nào đó.
\item So sánh ưu, nhược điểm của 2 phương pháp chính: lọc dựa trên nội dung (content-based filtering) và lọc cộng tác (Collaborative Filtering).
\item Dựa trên dữ liệu có sẵn (hiện tại không phải của công ty Zalora), hiện thực mức cơ bản sử dụng ngôn ngữ Python.
\end{itemize}

\subsection{Giai đoạn Luận Văn Tốt Nghiệp}
Mục tiêu của giai đoạn Luận Văn Tốt Nghiệp như sau:
\begin{itemize}
\item Xây dựng một hệ thống tư vấn với dữ liệu thật của công ty Zalora.
\item Áp dụng những phương pháp đã tìm hiểu trong giai đoạn Thực Tập Tốt Nghiệp với dữ liệu thật của công ty Zalora và so sánh để tìm được phương pháp có kết quả tốt nhất.
\item Từ phương pháp trả về kết quả tốt nhất, tập trung phân tích sâu hơn về phương pháp này.

\end{itemize}

\section{Phạm vi đề tài}
Tập trung vào hai phương pháp để tư vấn cho người dùng: Content-based Filtering và Collaborative Filtering.
Ngoài ra tìm hiểu thêm: Social Recommendation Systems. 
Hệ thống chỉ áp dụng trong công ty Zalora Việt Nam.
Các thông tin và sản phẩm tất cả đều từ dữ liệu của công ty.

\section{Cấu trúc báo cáo}
%%%%%%%%%%%%%%%%NOT FINISH%%%%%%%%%%%

%%%%%%%%%%%%%%%%%%%%%%%%%%%%%%%%%%%%%
%
%	End Chaper
%

\chapter{Phân tích vấn đề}

Trong chương này, Tôi sẽ trình bày các vấn đề đi kèm với các ví dụ cụ thể và phân tích, đưa ra giải pháp đề nghị phù hợp với giải pháp này.

\section{Ví dụ minh họa và phân tích}
Khi người dùng truy cập vào trang web thương mại điện tử, những vấn đề gây khó khăn cho người dùng khi trong việc tìm kiếm sản phẩm:
Số lượng thông tin, sản phẩm khá lớn, phong phú. Vì thế sẽ khiến người dùng gặp khó khăn khi xác định được sản phẩm người dùng mong muốn. Ví dụ: khi người dùng tìm kiếm từ jeans, hệ thống của website sẽ lọc và trả về những mặt hàng về jeans. Nhưng jeans thì lại có rất nhiều thể loại khác nhau, điển hình là jeans có thể chia theo giới tính.
\begin{figure}[h]
\centering
%\includegraphics[width=1.0\textwidth]{image/BepumExample.png}
\caption{Hình 2.1.1: Quần jeans dành phái nam1 và nữ2}
\end{figure}

Khi người dùng tìm kiếm một sản phẩm, thông tin nào đó, có thể người dùng muốn nhiều lựa chọn hơn hoặc sản phẩm, thông tin đó có phù hợp với người dùng hay không?
\begin{figure}[h]
\centering
%\includegraphics[width=1.0\textwidth]{image/BepumExample.png}
\caption{Hình 2.1.2: Hiển thị thông tin sản phẩm3}
\end{figure}

\section{Giải pháp đề nghị}
\subsection{Hệ thống tư vấn}
Có hai phương pháp phổ biến và được sử dụng rộng rãi, những phương pháp này sẽ được trình bày cụ thể hơn ở Chương 3: Lọc dựa trên nội dung (Content-based Filtering), Lọc cộng tác (Collaborative Filtering), Hybrid (kết hợp hai phương pháp trên).
Hệ thống cần phải xây dựng hồ sơ người dùng (user profile), cần lưu vết của mỗi người dùng khi họ tìm kiếm các từ khóa hoặc thao tác với sản phẩm nào đó. Những thông tin về lịch sử tìm kiếm, đặt/mua sản phẩm. Một yếu tố cũng có thể được quan tâm đến là thời gian khách hàng truy cập và tìm kiếm sản phẩm, vì đây có thể là thói quen của một số khách hàng.

\subsection{Cơ chế chọn quảng cáo phù hợp với khách hàng}
Ở giai đoạn đầu của quá trình hiện thực hệ thống, tôi sẽ sử dụng các thông tin sau: Lịch sử mua sắm của khách hàng. Xếp hạng (rating) về sản phẩm của khách hàng. Danh sách sản phẩm ưa thích (favorite List), mong muốn (wish List) của khách hàng.
Tất cả các thông tin này đều do dữ liệu của website cung cấp trong cơ sở dữ liệu và được hệ thống xử lý, khai phá. Từ đó hiển thị sản phẩm phù hợp với khách hàng.
Ở giai đoạn tiếp theo, tôi sẽ cân nhắc sử dụng thêm một số thông tin khác như: Thông tin cá nhân (giới tính, tuổi, nghề nghiệp…) từ mạng xã hội của khách hàng (có thể sẽ sử dụng thêm phương pháp Social Recommendation Systems).
Các thông tin ngữ cảnh khác ngoài vị trí khách hàng như thời gian hiện tại (gần ngày lễ tình yêu, quốc tế phụ nữ…). Các ghi chú, danh sách mua sắm (Shopping List) của khách hàng (nếu khách hàng cho phép).
Lưu ý là những thông tin này sẽ được tôi cân nhắc kỹ và chưa chắc sẽ được sử dụng nếu nguồn lực và thời gian không cho phép.
\section{Kiến trúc hệ thống}
Dưới đây là mô hình tổng quan về kiến trúc hệ thống

\begin{figure}[h]
\centering
%\includegraphics[width=0.8\textwidth]{image/SelfModifying.PNG}
\caption{Hình 2.3: Sơ đồ kiến trúc hệ thống}
\end{figure}

Front-end: người dùng sẽ tương tác với trang web thông qua việc tìm kiếm hoặc click chọn một sản phẩm mà người dùng ưu thích, dữ liệu này sẽ được chuyển đến khối Adapter để chuẩn hóa dữ liệu thành một dạng dữ liệu đầu vào chung cho mọi định dạng. Kết quả trả về từ Adapter sẽ là thông tin dữ liệu đã được chuẩn hóa theo định dạng đầu vào của Back-end. Nhiệm vụ của khối Back-end chính là thông qua hệ thống tư vấn sẽ trả về kết quả danh sách những sản phẩm được tư vấn thông qua giải thuật được sử dụng trong hệ thống tư vấn. Danh sách những sản phẩm này sẽ được chuẩn hóa trở lại thành định dạng ban đầu của nó. Cuối cùng sau khi chuẩn hóa, phía Front-end có nhiệm vụ hiển thị danh sách những sản phẩm đã được tư vấn.
Dưới đây là mô hình kiến trúc hệ thống tư vấn:
\begin{figure}[h]
\centering
%\includegraphics[width=0.8\textwidth]{image/SelfModifying.PNG}
\caption{Bảng 2.4: Thiết kế hệ thống tư vấn}
\end{figure}

Adapter Pattern: do dữ liệu đầu vào có thể là nhiều file extension khác nhau như: .csv, .sql…
Khi dữ liệu qua Adapter Pattern, mọi dữ liệu đều được xử lý theo 1 định dạng chung cho mọi dữ liệu đầu vào.

\chapter{Kiến thức nền tảng}
Trong chương này, Tôi sẽ giới thiệu chi tiết về hai phương pháp đã nêu ở trên, nền tảng trong hệ thống của tôi. 
\part{Kiến thức nền tảng}
\section{Lọc dựa trên nội dung (CBF: Content-based Filtering)}
Dữ liệu do người dùng cung cấp, có thể rõ ràng (được đánh giá) hoặc không rõ ràng. Dựa trên dữ liệu người dùng, hồ sơ người dùng được thiết lập, hồ sơ sẽ tư vấn cho người dùng. Sự tương tác, lịch sử tìm kiếm của người dùng càng nhiều với hệ thống, thì hệ thống sẽ tư vấn hiệu quả, chính xác và liên tục. 
Trong hệ thống lọc dựa trên nội dung, cần phải xây dựng cho mỗi mặt hàng một hồ sơ, hồ sơ này là một hoặc một tập hồ sơ đặc tả những tính đặc trưng quan trọng của mặt hàng. Trong một số trường hợp cụ thể, hồ sơ bao gồm các đặc trưng của mặt hàng mà nó dễ dàng được nhận thấy. Ví dụ, xem xét các đặc điểm của một bộ phim có thể liên quan đến hệ thống tư vấn:
\begin{itemize}
\item Tập các diễn viên trong bộ phim: sở thích của một số người dùng khi chọn xem một bộ phim nào đó có thể chỉ vì có diễn viên nổi tiếng và/hoặc người dùng yêu thích.
\item Đạo diễn: một số đạo diễn được yêu thích do cốt truyện rành mạch, các yếu tố bất ngờ của bộ phim khiến người dùng.
\item Năm của bộ phim được sản xuất: những bộ phim cũ (thông thường những bộ phim phiên bản đầu tiên thường lột tả được chính xác nội dung bộ phim), hoặc những bộ phim mới nhất tại thời điểm của họ vì hiệu ứng và công nghệ sẽ làm bộ phim thu hút người dùng.
\item Thể loại của bộ phim: khi người dùng có quá nhiều bộ phim, họ thường sẽ dựa trên những thể loại họ yêu thích (hài hước, hành động, kinh dị….).
\end{itemize}

\subsection{Ưu điểm và Nhược điểm}
Ưu điểm: [1]	Hệ thống có thể đưa ra tư vấn tốt nhất cho mỗi người dùng độc lập. Yêu cầu tính phân loại thấp, bởi vì mô hình người dùng có thể được tạo tự động.
Nhược điểm: [1]   Yêu cầu tính toán cao. Ví dụ: mỗi mặt hàng phải được phân tích bởi tính đặc trưng của nó, mô hình người dùng được xây dựng. Mô tả nội dung: Việc mô tả nội dung khó khăn. Ví dụ: Video, Music…Phụ thuộc tính năng của mặt hàng. Bỏ qua chất lượng và tính phổ biến của mặt hàng.

\subsection{Cốt lõi sử dụng trong Lọc dựa trên nội dung}
Sử dụng cơ chế TF-IDF [3] và Vector Space Model [4], ta có thể xác định được độ tương tự giữa hai document (tất cả những thông tin liên quan đến mặt hàng) được xét. [2]:
\subsubsection{Sử dụng cơ chế Term Frequency (TF)and Inverse Document Frequency(IDF)}
Thông qua các đặc điểm của document, để quyết định được quan hệ của một document, sử dụng cơ chế TF-IDF. Nói cách khác, độ tương tự (tương đồng) được định nghĩa là khoảng cách giữa các điểm, hoặc là góc giữa những vector trong không gian n-chiều:Làm thế nào để tính TF-IDF? Thông qua ví dụ dưới đây: Giả sử tập của tất cả documents là $10^6$. Mỗi DF, tổng các từ trong mỗi document. Ví dụ: DF của Analytics: 5000, Data: 50000…

\begin{figure}[h]
\centering
%\includegraphics[width=0.8\textwidth]{image/SelfModifying.PNG}
\caption{3.1: Dữ liệu của IoT and Analytics.}
\end{figure}

\begin{figure}[h]
\centering
%\includegraphics[width=0.8\textwidth]{image/SelfModifying.PNG}
\caption{Hình 3.2: Công thức tính TF}
\end{figure}

Dựa vào công thức hình 3.2, ta có bảng sau:
\begin{figure}[h]
\centering
%\includegraphics[width=0.8\textwidth]{image/SelfModifying.PNG}
\caption{Hình 3.3: Kết quả TF}
\end{figure}

Ví dụ: TF của (Article 1 – Analytics): \par
1 + $\lg tf _{Article 1– Analytics}$ = 1 + $\lg 21 _{Article 1– Analytics}$ \par

Giá trị của (Article 1 - Clound) = 0 do giá trị tại vị trí này nhỏ hơn hoặc bằng 0.
Để tính độ lớn của vector (Length of Vector), ta tính root-mean-squared-error:
Ví dụ: giá trị của hàng Article 1 với tất cả các cột (Analytics, Data, Cloud, Smart, Insight):\par
$\sqrt{2.322219295^2 + 2.380211242^2 + 0^2 + 1.301029996^2 + 1.301029996^2} = 3.800456039$ \par

Để tính IDF, ta tính $\lg \frac{N}{DF}$ với N: total docs là 1 triệu ($10^6$). Ta có:
\begin{figure}[h]
\centering
%\includegraphics[width=0.8\textwidth]{image/SelfModifying.PNG}
\caption{Hình 3.4: Kết quả IDF}
\end{figure}
Ví dụ: IDF của (Article 1 – Analytics): \par
$\lg \frac{10^6}{DF _{Article 1 – Analytics}} = \lg \frac{10^6}{5000} = 2.301029996$ \par


Để chuẩn hóa vector, độ lớn document vector / term vector.
\begin{figure}[h]
\centering
%\includegraphics[width=0.8\textwidth]{image/SelfModifying.PNG}
\caption{Hình 3.5: Sau khi chuẩn hóa}
\end{figure}

Ví dụ: chuẩn hóa giá trị (Article 1 – Analytics): \par
$\frac{TF _{Article 1 – Analytics}}{Length of vector _{Article}} = \frac{2.322219295}{3.800456039} = 0.61103701$ \par

Để xác định độ tương tự giữa hai article, tính cosine của 2 article đó, hay còn gọi là sum of dot product:
\begin{figure}[h]
\centering
%\includegraphics[width=0.8\textwidth]{image/SelfModifying.PNG}
\caption{Hình 3.6: Kết quả độ tương tự}
\end{figure}

Ví dụ: cos (Article 1, Article 2) =	0.611*0.594 + 0.626*0.692 + 0*0.325 = 0.796 = cos($A_1, A_2$)

Kết luận: ta sẽ chọn giá trị cosin nào gần 1 nhất vì như thế thì góc được tạo bởi 2 Article đó càng về 0. Từ đó ta có thể nói rằng 2 Article đó càng tương tự nhau.
\subsubsection{Sử dụng model Vector Space Model}
Mô hình không gian vector (Vector Space Model) [4]:  là một mô hình đại số (algebraic model) thể hiện thông tin văn bản như một vector, các phần tử của vector này thể hiện mức độ quan trọng của một từ và cả sự xuất hiện hay không xuất hiện (Bag of words) của nó trong một document.
Mỗi mặt hàng được xem như là một vector, trong đó gồm n thuộc tính của vector (thuộc tính cũng được xem là vector) n-chiều. Như vậy góc giữa các vector được tính toán để xác định độ tương tự giữa các vector. 
Vector hồ sơ người dùng sẽ được tạo dựa trên hành động trên các mặt hàng đã tìm kiếm ở quá khứ
\begin{figure}[h]
\centering
%\includegraphics[width=0.8\textwidth]{image/SelfModifying.PNG}
\caption{Hình 3.7: hai thuộc tính vector của một mặt hàng}
\end{figure}

Hình 3.6 biểu diễn 2 thuộc tính trong không gian 2-D (2 chiều), Cloud và Analytics. $M_1$, $M_2$ là thuộc tính. $U_1$, $U_2$ là người dùng. Khi giá trị của Analytics tăng thì thuộc tính $M_2$ tăng nhanh hơn so với $M_1$, và ngược lại, Khi giá trị của Cloud tăng thì thuộc tính $M_1$ tăng nhanh hơn so với $M_2$. U1 thích thuộc tính Cloud hơn so với Analytics, và $U_2$ ngược với $U_1$. Từ đó ta có thể xét cosin góc giữa vector hồ sơ người dùng $U_i$và vector thuộc tính.


\subsubsection{Lọc cộng tác (CF: Collaborative Filtering).}
Thay vì sử dụng các đặc điểm của các mặt hàng để đưa ra quyết định về độ tương tự của mặt hàng, ta tập trung vào độ tương tự của 2 mặt hàng nào đó được người dùng xếp hạng. 
Lọc cộng tác là một giải thuật phổ biến trong hệ thống, dựa vào dự đoán và tư vấn trên đánh giá hoặc thao tác của người dùng trong hệ thống. Phân tích mối liên hệ giữa người dùng và sự phụ thuộc lẫn nhau giữa sản phẩm và người dùng mới.
Giả sử nếu người dùng ưa thích những mặt hàng thông qua chất lượng, các thuộc tính đặc trưng…, người dùng có khả năng sẽ thích những mặt hàng khác. Ví dụ: một nhóm người cùng thích một mặt hàng nào đó, nếu John là thành viên trong nhóm đó, có khả năng John cũng sẽ thích mặt hàng đó.
Lọc cộng tác gồm neighborhood methods và mô hình yếu tố ngầm (latent factor):
\begin{itemize}
\item Neighborhood methods: được đánh giá theo trọng số trung bình tổng thể của rating trên các mặt hàng.
\item Latent factor: giải thích rating bằng đặc tả cả các mặt hàng lẫn người dùng.
\end{itemize}

\subsection{Ưu điểm và nhược điểm}
Ưu điểm [5]:   Lọc cộng tác có nội dung độc lập, không giới hạn tính toán trong khi xử lý. Bởi vì rating do người dùng bình chọn, dữ liệu được đánh giá thật, đảm bảo chất lượng. Thời gian tính toán của lọc cộng tác nhanh hơn lọc dựa trên nội dung.
Nhược điểm [5]:  vấn đề “cold-start”:  Nếu người dùng mới không đánh giá hoặc không có mặt hàng, hệ thống không thể tìm thấy những người dùng có cùng sở thích. Nếu mặt hàng là mới trong hệ thống và chưa được đánh giá. Trong quan hệ mới, không có người dùng nào đánh giá mặt hàng. Dữ liệu thưa thớt => không thể tư vấn.	

\subsection{Các phương pháp}
\subsubsection{Matrix Factorization (MF) [6]}
Mô hình yếu tố ngầm (latent factor model) sử dụng Matrix Factorization.
Đặc tả các mặt hàng và người dùng thông qua những vector yếu tố, được nội suy từ các mẫu xếp hạng mặt hàng.
Ưu điểm: Khả năng mở rộng tốt với dự đoán chính xác. Linh hoạt với nhiều mô hình.
Khi phản hồi không rõ ràng, hệ thống tư vấn có thể nội suy sở thích người dùng thông qua việc quan sát thao tác của người dùng: Lịch sử thanh toán, lịch sử duyệt web, mẫu tìm kiếm, những di chuyển của chuột.
Nhược điểm: đối với dữ liệu nhỏ (thông thường dữ liệu của các công ty nhỏ hoặc dữ liệu được sử dụng rộng rãi (public data)) sẽ trả về kết quả tốt và nhanh chóng. Tuy nhiên, với dữ liệu lớn, như của các công ty thương mại điện tử, kết quả trả về không hiệu quả, thậm chí thời gian đưa ra kết quả không đảm bảo với yêu cầu phi chức năng của người dùng thường là tối đa 5 giây.
Biểu thức toán học của Matrix Factorization:
Để miêu tả sự tương tác giữa người dùng u và mặt hàng i – sở thích chung của người dùng với tính đặc trưng của mặt hàng, qua công thức:	\par
$\hat{r} = q_i^T p_u$	(1)

Với $q_i$: mặt hàng i đại diện cho sự liên kết giữa mặt hàng và những đặc tính của mặt hàng đó, $p_u$: người dùng u đại diện cho sự liên kết giữa người dùng u và những đặc tính của mặt hàng đó.
Mục tiêu chính của cách tiếp cận này chính là tìm mối liên hệ giữa mỗi cặp mặt hàng và người dùng, từ đó sẽ tạo thành vector yếu tố $q_i$, $p_u$ $\in$ $R^f$. Khi đã có những vector yếu tố này, công việc đánh giá xếp hạng của người dùng dành cho item đó thông qua công thức (1).
Mô hình tiêu biểu trong lọc công tác thường được sử dụng đó là SVD (Singular Value Decomposition). Mô hình này yêu cầu ma trận, trong đó mỗi giá trị của ma trận là giá trị rating của người dùng cho mặt hàng đó. Nhưng vấn đề khó khăn khi sử dụng mô hình này là khi khá nhiều giá trị trong ma trận missing (nghĩa là những mặt hàng đó không được người dùng đánh giá).  
Để tìm các vector yếu tố ($p_u$, $q_i$), hệ thống tối thiểu cost function:		\par

$min_{q^*, p^*} \sum_{(u,i \in K)} ({r_{ui} - q_i^T p_u}^2) + \lambda ({{\parallel q_i \parallel}^2 + {\parallel p_u \parallel}^2}) $  (2)\par

$\lambda$ cross-validation: chia dữ liệu thành k tập con có cùng kích thước. Tại mỗi tập, ta xét 1 tập con trong tập đó là tập test, các tập còn lại là tập training.
Có 2 cách tiếp cận để là tối thiểu hóa giá trị của công thức (2), đó là Stochastic Gradient Descent (SGD) và Alternating Least Squares (ALS):
\begin{itemize}
\item SGD: Với mỗi trường hợp trong tập training, hệ thống sẽ dự đoán rui và kiểm tra giá trị dự đoán chênh lệch bao nhiêu so với giá trị chính xác được tính ra từ trường hợp đó trong tập training:
\end{itemize} \par

\tab $e_{ui} = \hat{r} = q_i^T p_u $	\par


\tab Sau đó, nhiệm vụ tiếp theo là 	thay đổi các giá trị của ($q_i, p_u$) dựa trên các \tab giá trị kiểm tra. Công việc này sẽ dùng vòng lặp cho tới khi giá trị kiểm tra \tab nhỏ nhất và giá trị của ($q_i, p_u$) phù hợp nhất.
\begin{itemize}
\item ALS: một cách tiếp cận khác là sử dụng Alternating Least Squares. Trong kỹ thuật ALS, khi ta cố định tất cả các giá trị (ví dụ $p_u$), hệ thống sẽ tính $q_i$ thông qua xử lý vấn đề least-squares, hoặc ngược lại.\par
Trong khi SGD dễ dàng sử dụng và hiện thực và thời gian xử lý nhanh hơn so với ALS, nhưng ALS lại được ưu chuộng hơn khi sử dụng vì 2 lý do: bởi vì 1 trong 2 giá trị ($q_i, p_u$) đã được cố định, vì thế hệ thống sẽ chỉ cần tính hoặc $q_i$ hoặc $p_u$ độc lập với nhau. Thứ hai là hiệu quả trong khi tập dữ liệu không rõ ràng, cụ thể. 
\end{itemize}
Thêm Bias: ta có thể hiểu đơn giản bias khi được thêm vào, nó sẽ ảnh hưởng đến mối liên hệ giữa user và item. Ví dụ: trong lọc cộng tác, dữ liệu trong hệ thống lớn có khuynh hướng là một số user đánh giá các item cao hơn so với người khác.

\tab $q_i \leftarrow q_i + \gamma(e_{ui}p_{u} - \lambda q_i)$ \par
\tab $p_u \leftarrow p_u + \gamma(e_{ui}q_{i} - \lambda p_u)$ \par


Trong đó bui là bias ảnh hưởng đến giá trị xếp hạng rui, bu, bi: độ quan sát của người dùng và mặt hàng từ $\mu$ với $\mu$: xếp hạng trung bình tổng thể. 
Từ đó, ta có thể mở rộng cách tính tối thiểu cost function – squared error function: \par 
\tab $\hat{r_{ui}}(t) = \mu + b_i(t) +b_u(t) + q_i^T p_u(t)$ \par

Để làm giảm trọng số quan sát, ta có thể thêm mức độ tin tưởng cui vào công thức (4):	 \par

$min_{p*, q*, b*}\sum_{(u,i)\in K} c_{ui}(r_{ui} - \mu - b_u - b_i - q_i^T p_u)^2 + \lambda(\parallel p_u \parallel^2 + \parallel q_i \parallel^2 + b_u^2 + b_i^2)$ \par

\begin{figure}[h]
\centering
%\includegraphics[width=0.8\textwidth]{image/SelfModifying.PNG}
\caption{Hình 3.8: Minh họa phân rã ma trận}
\end{figure}		\par

Để dự đoán xếp hạng của người dùng hàng 2, cột 2, ta áp dụng biểu thức (1), ta có: \par 
$R_{ui}$ = 0.5*0.8 + 0.6*0.1 = 0.46 \par

\subsubsection{Factorization Machines [8]}
Với những tập dữ liệu nhỏ, việc sử dụng matrix factorization để xử lý những vấn đề missing rating sẽ nhận được những kết quả tích cực. Nhưng đối với tập dữ liệu lớn (Ví dụ: tập dữ liệu của các công ty thương mại điện tử), lượng dữ liệu cần được xử lý là rất lớn, kết quả khi sử dụng matrix factorization không đạt được hiệu quả như mong đợi. Vì vậy một cách tiếp cận khác để xử lý vấn đề dữ liệu thưa thớt lớn (missing value). Đó là Factorization Machines, nó sẽ kết hợp những ưu điểm của Support Vector Machine (SVM) và mô hình phân rã ma trận (factorization model). \par
Điểm giống với SVM đó là đều tính toán trên những giá trị thực của các vector yếu tố. Nhưng trái với SVM, FMs mô hình hóa tất cả sự tương tác giữa các biến được sử dụng trong các tham số được phân rã => có khả năng đánh giá sự tương tác lẫn nhau, thậm chí trong vấn đề dữ liệu thưa thớt lớn mà SVM không giải quyết được. Và ưu điểm khác của FMs đó là thời gian tính toán là tuyến tính, như vậy độ phức tạp chỉ là O(n) => Tối ưu hóa vấn đề. \par
Ưu điểm:  Cho phép đánh giá tham số khi dữ liệu thưa thớt. Độ phức tạp là tuyến tính, được tối ưu. Làm việc với các giá trị thực của vector đặc trưng. \par

\begin{figure}[h]
\centering
%\includegraphics[width=0.8\textwidth]{image/SelfModifying.PNG}
\caption{Hình 3.9: Ví dụ cho những giá trị rời rạc thực của vector đặc trưng x}
\end{figure}		\par 

Mỗi hàng biểu diễn vector đặc trưng $x^(i)$ với mục tiêu trả về là $y^(i)$. 
Theo như hình 3.7, hầu hết các giá trị của vector đặc trưng x đều bằng 0. Đặt m(x) là số lượng những phần tử của vector đặc trưng x bằng 0 và mD là trung bình số lượng những phần tử còn lại khác 0 trong vector đặc trưng x. \par 

Biểu thức toán học của Factorization Machines: \par
\begin{itemize}
\item Trường hợp 2 chiều, biểu thức được biểu diễn như sau:
\end{itemize} \par

\tab $\hat{y}(x) := w_0 + \sum_{i=1}^{n}w_ix_i + \sum_{i=1}^{n}\sum_{i=i+1}^{n}\langle v_i,v_j \rangle x_i x_j$		(1)\par
Với: \tab $\langle v_i,v_j \rangle := sum_{f=1}^{k} v_{i,f} v_{j,f}$ \par 
$w_0$:  bias toàn cục, $w_i$: trọng số của biến thứ i. <$v_i, v_j$> mô hình hóa sự tương tác giữa biến thứ i và j để phân rã. \par
Với dữ liệu thưa thớt lớn sẽ không đủ dữ liệu để đánh giá sự tương tác giữa các biến một cách trực tiếp và độc lập. FMs có thể xử lý được vấn đề này bằng cách loại bỏ các tham số độc lập bằng cách phân rã chúng. Có nghĩa là một dữ liệu không missing có thể giúp ta đánh giá những dữ liệu missing khi những dữ liệu này có liên quan đến dữ liệu tương tác với chúng. \par

\tab $\sum_{i=1}^{n}\sum_{j=i+1}^{n}\langle v_i,v_j \rangle x_i x_j = \frac{1}{2}\sum_{f=1}^{k}((\sum_{i=1}^{n}v_{i,f}x_i x_j)^2 - \sum_{i=1}^{n}v_{i,f}^2 x_i^2)$ \par

Từ biểu thức trên, ta có thể thấy thời gian tính toán sẽ là O(kn). \par
Với tác vụ dự đoán, FMs có đủ khả năng iđược sử dụng trong 3 nhóm sau đây: \par Regression, Binary Classification, Ranking. Trong những trường hợp này, biểu thức chính quy sẽ được thêm vào để tránh vấn đề overfitting. \par
Để tối ưu hóa thời gian tính toán theo tuyến tính, các tham số chính $w_0$, w, V sẽ được tối ưu hóa thông qua phương thức Stochastic Gradient Descent (SGD), bằng cách đạo hàm biểu thức (1) theo các tham số nêu trên. Từ đó ta có biểu thức sau: \par

%%%%%%%%%%%%%%%%%%%%%%%NOT FINISH%%%%%%%%%%%%%%%%%%%%%%%%%%%%%%%
%%%%%%%%%%%%%%%%%%%%%%%%%%%%%%%%%%%%%%%%%%%%%%%%%%%%%%%%%%%%%%%%

Theo biểu thức trên, mỗi giá trị góc sẽ có thời gian tính toán là O (1) => toàn bộ thời gian tính toán là O(kn). Điều này đã chứng minh được FMs có thời gian tính toán là tuyến tính.	\par
Trường hợp d-chiều, biểu thức sẽ được mở rộng dựa trên biểu thức (1) như sau: \par
\tab $\hat{y}(x) := w_0 + \sum_{i=1}^{n}w_i x_i + \sum_{l=2}^{d}\sum_{i_1}^{n}\ldots\sum_{i_l=i_{l-1}+1}^{n}(\Pi_{j=1}^{l}x_{i_j})(\sum_{f=1}^{k_l}\Pi_{j=1}^{l}v_{i_{i,f}}^l)$ \tab (2)\par 
Nếu theo biểu thức (2), thời gian tính toán sẽ trở thành O ($k_d, n^d$).

\part{Công nghệ sử dụng}
Hiện nay có rất nhiều ngôn ngữ hỗ trợ rất tốt trong lĩnh vực Data Science, nhưng phổ biến và được ưu chuộng nhất hiện nay có thể là Python, R, và C++ (ngoài ra còn có Matlab vì đây cũng là một công cụ khá mạnh và hiệu quả khi ta cần xử lý ảnh hoặc cần tìm những điểm nổi bật khi thống kê số liệu để đưa ra những kết luận đặc trưng). Những ngôn ngữ này hỗ trợ rất nhiều thư viện để xử lý vấn đề trong Machine Learning, DataMining, Image Processing, Recommendation Systems…		\par
Nhưng trong bài báo cáo này, tôi tập trung sử dụng ngôn ngữ Python vì: dễ dàng sử dụng, cú pháp đơn giản, ngắn gọn, cụ thể, thư viện hỗ trợ khá phong phú. Python được ra đời lần đầy tiên vào năm 1991 bởi Guido van Rossom. Python được sử dụng rộng rãi trên toàn thế giới và là ngôn ngữ được các trường đại học tại Mỹ ưu dùng để giảng dạy. Python được hỗ trợ tốt trên cả ba hệ điều hành phổ biến nhất hiện nay là Windows, Linux, MacOS, dễ dàng cài đặt và sử dụng.Ngoài ra JetBrain còn đưa ra IDE được dùng cho Python, đó là PyCharm. Python còn hỗ trợ lập trình hướng đối tượng (OOP-Oriented Object Programming). \par 
Các thư viện hỗ trợ trong Machine Learning, đó là: Numpy, Scipy, Pandas, Matplotlib, Scikit-learn…Và trong báo cáo này, tôi đều sử dụng tất cả các thư viện vừa nêu trên.	\par
Ngoài ra, đối với Matrix Fatorization, ta có thể cài đặt gói package của Python tên là PyMF,đối với Factorization Machines, fastFM là thư viện khá tốt để hiện thực bằng Python.	\par
%
%	End Chaper
%

\chapter{KẾT QUẢ}

Trong chương này, tôi sẽ tóm tắt kết quả mà tôi đạt được trong giai đoạn Thực Tập Tốt Nghiệp. Cuối cùng là phần kế hoạch của tôi cho giai đoạn Luận Văn Tốt Nghiệp.

\section{Kết quả đạt được trong giai đoạn Thực Tập Tốt Nghiệp}
\subsection{Tóm tắt kết quả}
Trong giai đoạn Thực Tập Tốt Nghiệp, tôi đã đạt được các kết quả sau:
\begin{itemize}
\item Phân tích tính thực tiễn, xác định hướng đi tốt nhất cho hệ thống.
\item Tìm hiểu về Hệ thống tư vấn, những vấn đề cần giải quyết trong hệ thống tư vấn.
\item Các phương pháp sử dụng trong hệ thống.
\item Hiện thực trên dữ liệu thật tìm trên mạng.
\end{itemize} \par
Bên cạnh đó, quá trình làm việc của tôi từ lúc bắt đầu đề tài đến lúc kết thúc giai đoạn Thực Tập Tốt Nghiệp cũng được đính kèm ở Phụ Lục A. \par

\subsection{Kế hoạch cho giai đoạn Luận Văn Tốt Nghiệp}
\begin{center}
\captionof{table}{Trình bày kế hoạch của tôi cho giai đoạn Luận Văn Tốt Nghiệp.} 
\begin{tabular} { |c|c|c|c|c| } 
\hline
Thời gian & Công việc dự kiến \\
\hline
06/2017 - 09/2017 & Xin dữ liệu thật từ công ty Zalora, trụ sở chính ở Singapore. \\
& Sau khi nhận được dữ liệu thật, tiến hành hiện thực các phương pháp trên vào dữ liệu thật của công ty. \\
%\begin{itemize}
%\item Xin dữ liệu thật từ công ty Zalora, trụ sở chính ở Singapore.
%\item Sau khi nhận được dữ liệu thật, tiến hành hiện thực các phương pháp trên vào dữ liệu thật của công ty.
%\item Hiện thực Adapter Pattern cho hệ thống sẽ thiết lập.
%
%\end{itemize} \\
%\hline
%B & 4.99 & 4 \\
%\hline
%C & 5.0 & 3 \\
%\hline
%D & 4.98 & 5 \\
%\hline
%E & 5.2 & 1 \\
%\hline
\end{tabular}
\end{center}






%
%	End Chaper
%


\newpage
\bibliographystyle{plain}
\bibliography{thesis_reference.bib}

\end{document}